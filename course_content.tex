\sectionCenteredToc{Введение}

В эпоху цифровой трансформации производительность центральных
процессоров становится ключевым фактором эффективности
вычислительных систем. Развитие технологий приводит к постоянному
появлению новых архитектурных решений, направленных на повышение
производительности и энергоэффективности.

Вследствие этого возникает многообразие архитектур, использующих
различные подходы к построению многоядерных систем. Понимание того, как
эти подходы влияют на производительность при решении конкретных
вычислительных задач, представляет значительный научный и практический
интерес.

Актуальность темы данной курсовой работы обусловлена несколькими
ключевыми факторами. Выбранные для сравнения процессоры
являются популярными представителями своих линеек в сегменте производительных мобильных
компьютеров и построены на принципиально разных архитектурных подходах.
Их прямое сравнение позволяет на конкретном примере оценить преимущества
и недостатки современных тенденций в проектировании центральных процессоров.

В качестве тестовой нагрузки выбрано быстрое преобразование Фурье (БПФ).
Это вычислительно интенсивная задача, эффективность выполнения которой 
напрямую зависит от архитектурных особенностей процессора. Поэтому анализ 
производительности БПФ является репрезентативным тестом для широкого класса 
научных и инженерных приложений.

Целью данной курсовой работы является проведение сравнительного анализа
производительности процессоров \textit{Intel Core i5-12450H} и \textit{AMD Ryzen 7 5800H}
при выполнении алгоритмов преобразования Фурье.

Для достижения поставленной цели необходимо решить следующие задачи:

\begin{enumerate_num_without_dot}
    \item Изучение технических характеристик и архитектурных особенностей
    рассматриваемых процессоров.
    \item Разработка программы для выполнения преобразования Фурье.
    \item Проведение вычислительных экспериментов с различными размерами
    данных, числом ядер и потоков.
    \item Анализ полученных данных для оценки производительности процессоров.
\end{enumerate_num_without_dot}

В результате исследования будут построены графики, демонстрирующие
зависимость времени выполнения преобразования Фурье от размера данных,
что позволит провести детальный анализ производительности
рассматриваемых процессоров.


\section{Архитектура вычислительной системы}
\subsection{Структура и архитектура вычислительной системы}

\subsection{История, версии и достоинтсва}

\subsection{Обоснование выбора вычислительной системы}

\subsection{Анализ выбранной вычислительной системы для написания программы}
