%! TeX root = main.tex

\sectionCenteredToc{Введение}

В эпоху цифровой трансформации производительность центральных
процессоров становится ключевым фактором эффективности
вычислительных систем. Развитие технологий приводит к постоянному
появлению новых архитектурных решений, направленных на повышение
производительности и энергоэффективности.

Вследствие этого возникает многообразие архитектур, использующих
различные подходы к построению многоядерных систем. Понимание того, как
эти подходы влияют на производительность при решении конкретных
вычислительных задач, представляет значительный научный и практический
интерес.

Актуальность темы данной курсовой работы обусловлена несколькими
ключевыми факторами. Выбранные для сравнения процессоры
являются популярными представителями своих линеек в сегменте производительных ноутбуков 
и построены на принципиально разных архитектурных подходах.
Их прямое сравнение позволяет на конкретном примере оценить преимущества
и недостатки современных тенденций в проектировании центральных процессоров.

В качестве тестовой нагрузки выбрано быстрое преобразование Фурье (БПФ).
Это вычислительно интенсивная задача, эффективность выполнения которой 
напрямую зависит от архитектурных особенностей процессора. Поэтому анализ 
производительности БПФ является репрезентативным тестом для широкого класса 
научных и инженерных приложений.

Целью данной курсовой работы является проведение сравнительного анализа
производительности процессоров \textit{Intel Core i5-12450H} и \textit{AMD Ryzen 7 5800H}
при выполнении алгоритмов преобразования Фурье.

Для достижения поставленной цели необходимо решить следующие задачи:

\begin{enumerate_num_without_dot}
    \item Изучение технических характеристик и архитектурных особенностей
    рассматриваемых процессоров.
    \item Разработка программы для выполнения преобразования Фурье.
    \item Проведение вычислительных экспериментов с различными размерами
    данных, числом ядер и потоков.
    \item Анализ полученных данных для оценки производительности процессоров.
\end{enumerate_num_without_dot}

В результате исследования будут построены графики, демонстрирующие
зависимость времени выполнения преобразования Фурье от размера данных,
что позволит провести детальный анализ производительности
рассматриваемых процессоров.

\section{Архитектура вычислительной системы}
\subsection{Структура и архитектура вычислительной системы}

Процессоры \textit{AMD Ryzen 7 5800H} и \textit{Intel Core i5-12450H} 
основаны на принципиально разных архитектурных подходах к построению 
вычислительных систем и ориентированы мобильные вычислительные системы.

\begin{table}[h!]
\centering
\caption{Сравнение характеристик}
\begin{tabularx}{\textwidth}{|c
                             |>{\centering\arraybackslash}m{5cm}
                             |>{\centering\arraybackslash}X|}
\hline
Характеристики & \textit{AMD Ryzen 7 5800H} & \textit{Intel Core i5-12450H} \\
\hline
Кодовое имя архитектуры & \textit{Cezanne-H (Zen 3)} & \textit{Alder Lake-H} \\
\hline
Физические ядра & 8 & 8 \\
\hline
Количество потоков & 16 & 12 \\
\hline
\textit{L1} кэш на ядро & 64 КБ & 80 КБ + 96 КБ \\
\hline
\textit{L2} кэш на ядро & 512 КБ & 1.25 МБ + 2 МБ\footnotemark[1] \\
\hline
\textit{L3} кэш & 16 МБ & 12 МБ \\
\hline
Тактовая частота & 3.2 - 4.4 Ггц & 2 - 4.4 Ггц \\
\hline
Техпроцесс & 7 нм & 10 нм \\
\hline
Встроенная графика & \textit{AMD Radeon RX Vega 8} & \textit{Intel UHD Graphics Xe G4} \\
\hline
Поддерживаемая память & DDR4 & DDR4, DDR5 \\
\hline
\textit{TDP} & 45 Вт & 45 Вт \\
\hline
\textit{Hyper-Threading/SMT} & + & + \\
\hline
\end{tabularx}
\end{table}

\footnotetext[1]{E-ядра используют общий 2 МБ кластер}

Процессор \textit{AMD Ryzen 7 5800H}, использует 
монолитную архитектуру \textit{Zen 3}, произведенную по техпроцессу 
7 нм \cite{specsRyzen}.  Данная архитектура предполагает использование 8 ядер, 
которые с помощью технологии \textit{SMT} (\textit{Simultaneous Multithreading}) 
обеспечивают обработку 16 потоков. Максимальная тактовая частота 
составляет 4.4 ГГц. Процессор оснащен 16 МБ кэш-памяти третьего 
уровня (\textit{L3}), а его тепловой пакет (TDP) составляет 45 Вт.
За обработку графики отвечает встроенный видеоадаптер \textit{AMD Radeon RX Vega 8}. 
Он построен на проверенной временем архитектуре \textit{Vega}, имеет в своем составе 8 
вычислительных блоков и способен работать на частоте до 2.0 ГГц.

Процессор \textit{Intel Core i5-12450H} базируется 
на гибридной архитектуре \textit{Alder Lake}, изготовленной по техпроцессу 
10 нм \cite{specsIntel}. В его состав входят 4 производительных \textit{P}-ядра и 4 энергоэффективных \textit{E}-ядра, 
что обеспечивает поддержку 12 потоков. Распределение нагрузки между ядрами 
осуществляется аппаратным планировщиком \textit{Intel Thread Director}. 
Пиковая частота производительных ядер достигает 4.4 ГГц, объем кэш-памяти \textit{L3} 
составляет 12 МБ, а тепловой пакет (\textit{TDP}) — 45 Вт.
В него интегрировано графическое ядро \textit{Intel UHD Graphics Xe G4}, 
принадлежащее к более современной архитектуре \textit{Xe-LP}. Оно включает 48 
исполнительных блоков, но работает на более низкой пиковой частоте в 1.2 ГГц.

Таким образом, ключевое различие между процессорами заключается в подходе к 
многоядерности. Архитектура \textit{Zen 3}, позволяющая процессору оперировать 16 потоками, потенциально лучше 
подходит для тяжелых параллельных вычислений, где 
важна максимальная производительность каждого потока, а увеличенный объем 
кэш-памяти L3 способствует ускорению работы с большими наборами данных. В 
свою очередь, архитектура \textit{Alder Lake} использует более современную 
модель с разнородными ядрами, которая обеспечивает гибкость и энергоэффективность 
за счет распределения задач между производительными и эффективными ядрами с помощью аппаратного 
планировщика \textit{Intel Thread Director}. Однако эффективность такого подхода 
сильно зависит от оптимизации операционной системы и программного обеспечения 
под гибридную архитектуру. Эти фундаментальные различия в дизайне и станут 
основой для последующего сравнительного анализа их производительности в 
реальных задачах.

\subsection{История, версии и достоинтсва}

Архитектура \textit{Zen 3}, на которой базируется \textit{Ryzen 7 5800H}, является кульминацией 
многолетних усовершенствований. Её история начинается с революционной архитектуры 
\textit{Zen} (2017), которая ознаменовала возвращение \textit{AMD} в сегмент 
высокопроизводительных процессоров. Последующая \textit{Zen 2} привнесла 7-нм 
техпроцесс и чиплетный дизайн, значительно увеличив количество ядер. 
\textit{Zen 3} стала её логическим развитием, сфокусированным на повышении 
показателя \textit{IPC} (числа инструкций за такт) и снижении задержек. 
Ключевым нововведением стало объединение ядер и кэш-памяти \textit{L3}
в единый комплекс, что улучшило межъядерное взаимодействие. 
Основные достоинства этой архитектуры заключаются в её монолитной 
структуре с восемью одинаково производительными ядрами, что обеспечивает 
предсказуемую производительность. Благодаря технологии \textit{SMT} процессор 
обрабатывает 16 потоков, что делает его отлично подходящим для параллельных 
вычислений, а большой объединенный \textit{L3} кэш ускоряет работу с крупными наборами данных.

В свою очередь, архитектура \textit{Alder Lake}, лежащая в основе 
\textit{Core i5-12450H}, знаменует собой самый радикальный сдвиг для \textit{Intel}
за последнее десятилетие. Ей предшествовал долгий период доминирования 
архитектуры \textit{Skylake} и её многочисленных итераций (от \textit{Kaby Lake} 
до \textit{Comet Lake}), производившихся по 14-нм техпроцессу. Столкнувшись 
с растущей конкуренцией, \textit{Intel} разработала \textit{Alder Lake} 
как комплексный ответ. Эта архитектура не только перешла на новый техпроцесс 
\textit{Intel 7} (10 нм), но и впервые в настольном сегменте представила гибридную 
модель. Она сочетает в себе производительные ядра (\textit{P-cores}) для выполнения 
сложных задач и энергоэффективные ядра (\textit{E-cores}) для фоновых процессов. 
Управление распределением задач между ними осуществляет аппаратный планировщик
\textit{Intel Thread Director}. Преимущества такого подхода заключаются в гибкости 
и энергоэффективности, поскольку система адаптируется к нагрузке, оптимизируя 
энергопотребление. \textit{P-ядра} обеспечивают высокую однопоточную производительность, 
а поддержка современных технологий, таких как память \textit{DDR5}, гарантирует 
высокую пропускную способность.

\subsection{Обоснование выбора вычислительной системы}

Выбор процессоров \textit{AMD Ryzen 7 5800H} и \textit{Intel Core i5-12450H} для сравнительного 
анализа обусловлен тем, что они, принадлежа к одному классу мобильных решений 
с одинаковым тепловым пакетом, основываются на принципиально разных архитектурных 
философиях. Их сопоставление позволяет наглядно оценить сильные и слабые стороны 
двух доминирующих подходов в современном процессоростроении: «классического» 
монолитного с однородными ядрами и нового гибридного с разнородными. Ключевым аспектом для 
сравнения является реализация технологий виртуальной многопоточности: \textit{SMT} (\textit{Simultaneous Multithreading}) 
у \textit{AMD} и \textit{Hyper-Threading} у \textit{Intel}. В процессоре \textit{Ryzen 7 5800H} технология \textit{SMT} применяется 
ко всем восьми однородным ядрам, удваивая количество потоков до 16. Это создает 
симметричную и предсказуемую среду для параллельных вычислений. В свою очередь, 
в \textit{Intel Core i5-12450H} технология \textit{Hyper-Threading} активна только на четырех производительных 
\textit{P}-ядрах, в то время как четыре энергоэффективных \textit{E}-ядра не поддерживают ее. В 
результате общее число потоков составляет 12 (8 от \textit{P}-ядер и 4 от \textit{E}-ядер), что 
создает асимметричную структуру. Именно анализ того, как операционная 
система и приложение для тестов будут распределять нагрузку в этих двух принципиально 
разных моделях многопоточности, и является одной из центральных задач исследования. 
Наконец, принадлежность процессоров к одному сегменту и схожий уровень энергопотребления 
создают равные условия для анализа именно архитектурных различий, что делает 
сравнение объективным.

Выбор данных моделей также обусловлен их широкой распространенностью на рынке, 
что делает исследование актуальным для многих пользователей. В качестве 
тестовой нагрузки было выбрано быстрое преобразование Фурье (БПФ) — фундаментальный 
алгоритм, чувствительный к производительности ядер и подсистемы памяти. Его 
параллельная природа позволяет эффективно оценить, как каждая архитектура 
справляется с масштабированием вычислительной нагрузки.


\subsection{Анализ выбранной вычислительной системы для написания программы}

Анализ производительности выбранных процессоров целесообразно начать с рассмотрения
их типичных результатов в популярных синтетических бенчмарках, таких как \textit{Cinebench R24}, 
\textit{Geekbench 5} и \textit{Geekbench 6} и \textit{PassMark}. Эти тесты являются индустриальным стандартом
для оценки производительности \textit{CPU} и позволяют сделать предварительные выводы о поведении
процессоров в различных сценариях.

\textit{Cinebench R24} — это бенчмарк, который оценивает производительность процессора 
путем рендеринга сложной трехмерной сцены \cite{CinebenchInfo}. Тест доступен в двух режимах: 
однопоточном, который измеряет производительность одного ядра, и многопоточном, 
который задействует все доступные процессорные потоки. Этот бенчмарк хорошо 
показывает пиковую производительность \textit{CPU} в задачах, которые эффективно 
распараллеливаются.

\textit{Geekbench 6} — это кросс-платформенный бенчмарк, который измеряет производительность 
системы в задачах, имитирующих реальные сценарии использования, от просмотра 
веб-страниц до обработки изображений и машинного обучения \cite{GeekbenchInfo}. Он также предоставляет 
результаты для однопоточного и многопоточного режимов, что позволяет составить 
комплексное представление о возможностях процессора.

Сравнивая результаты тестирования, можно отметить, что в многопоточном тесте Cinebench 
R24 процессор \textit{AMD Ryzen 7 5800H} опережает \textit{Intel Core i5-12450H} примерно на 20\%,
что показывает его эффективность при работе в многопотоке, в то время как 
в однопоточном режиме разница составляет 10\% в пользу \textit{Intel Core i5-12450H}, что подчеркивает 
его высокую производительность при выполнении задач с использованием одного ядра. 

Тест \textit{Geekbench 6} показывает довольно схожие результаты. В многопоточном тесте 
\textit{Ryzen 7 5800H} вырывается вперед на 16\%, оставляя позади \textit{Core i5-12450H}. 
Но в однопоточном режиме по прежнему лидирует \textit{Core i5-12450H} с отрывом в 12.5\%,
что доказывает его преимущество в задачах, не требующих использования нескольких ядер.

%NOTE: Перефразировать и укоротить текст снизу

Эти данные показывают, что **AMD Ryzen 7 5800H** значительно производительнее в задачах рендеринга, требующих максимальной многопоточной производительности. В то же время, **Intel Core i5-12450H** демонстрирует более высокую производительность как в однопоточных задачах, так и в смешанных многопоточных сценариях, представленных в Geekbench 5, что говорит о высокой эффективности его P-ядер и хорошей работе планировщика.

Основываясь на этих данных, можно сделать следующие предположения относительно выполнения БПФ:

\begin{itemize}
    \item \textbf{В однопоточном режиме:} Учитывая стабильное преимущество **Intel Core i5-12450H** в однопоточных тестах, можно ожидать, что при выполнении БПФ в одном потоке он покажет лучший результат.

    \item \textbf{В многопоточном режиме:} Ситуация неоднозначна. С одной стороны, преимущество **Ryzen 7 5800H** в Cinebench говорит о его высоком потенциале в чисто вычислительных, хорошо распараллеливаемых задачах, какой и является БПФ. С другой стороны, хороший результат **Core i5-12450H** в многопоточном Geekbench 5 указывает на то, что его гибридная архитектура может быть очень эффективной. Вероятно, результат будет зависеть от конкретной реализации алгоритма и того, насколько эффективно он сможет утилизировать все доступные потоки на обеих архитектурах.
\end{itemize}

\section{Платформа программного обеспечения}

