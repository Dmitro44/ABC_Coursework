%! TeX root = main.tex

\sectionCenteredToc{Введение}

В эпоху цифровой трансформации производительность центральных
процессоров становится ключевым фактором эффективности
вычислительных систем. Развитие технологий приводит к постоянному
появлению новых архитектурных решений, направленных на повышение
производительности и энергоэффективности.

Вследствие этого возникает многообразие архитектур, использующих
различные подходы к построению многоядерных систем. Понимание того, как
эти подходы влияют на производительность при решении конкретных
вычислительных задач, представляет значительный научный и практический
интерес.

Актуальность темы данной курсовой работы обусловлена несколькими
ключевыми факторами. Выбранные для сравнения процессоры
являются популярными представителями своих линеек в сегменте производительных ноутбуков 
и построены на принципиально разных архитектурных подходах.
Их прямое сравнение позволяет на конкретном примере оценить преимущества
и недостатки современных тенденций в проектировании центральных процессоров.

В качестве тестовой нагрузки выбрано быстрое преобразование Фурье (БПФ).
Это вычислительно интенсивная задача, эффективность выполнения которой 
напрямую зависит от архитектурных особенностей процессора. Поэтому анализ 
производительности БПФ является репрезентативным тестом для широкого класса 
научных и инженерных приложений.

Целью данной курсовой работы является проведение сравнительного анализа
производительности процессоров \textit{Intel Core i5-12450H} и \textit{AMD Ryzen 7 5800H}
при выполнении алгоритмов преобразования Фурье.

Для достижения поставленной цели необходимо решить следующие задачи:

\begin{enumerate_num_without_dot}
    \item Изучение технических характеристик и архитектурных особенностей
    рассматриваемых процессоров.
    \item Разработка программы для выполнения преобразования Фурье.
    \item Проведение вычислительных экспериментов с различными размерами
    данных, числом ядер и потоков.
    \item Анализ полученных данных для оценки производительности процессоров.
\end{enumerate_num_without_dot}

В результате исследования будут построены графики, демонстрирующие
зависимость времени выполнения преобразования Фурье от размера данных,
что позволит провести детальный анализ производительности
рассматриваемых процессоров.


\section{Архитектура вычислительной системы}
\subsection{Структура и архитектура вычислительной системы}

Процессоры \textit{AMD Ryzen 7 5800H} и \textit{Intel Core i5-12450H} 
основаны на принципиально разных архитектурных подходах к построению 
вычислительных систем и ориентированы мобильные вычислительные системы.

Процессор \textit{AMD Ryzen 7 5800H}, \cite{tpuProcessors} выпущенный в 2021 году, использует 
монолитную архитектуру \textit{Zen 3}, произведенную по техпроцессу 
7 нм. Данная архитектура предполагает использование 8 ядер, 
которые с помощью технологии \textit{SMT} (\textit{Simultaneous Multithreading}) 
обеспечивают обработку 16 потоков. Максимальная тактовая частота 
составляет 4.4 ГГц. Процессор оснащен 16 МБ кэш-памяти третьего 
уровня (\textit{L3}), а его тепловой пакет (TDP) составляет 45 Вт.

Процессор \textit{Intel Core i5-12450H} был выпущен в 2022 году и базируется 
на гибридной архитектуре \textit{Alder Lake}, изготовленной по техпроцессу 
7 нм. В его состав входят 4 производительных \textit{P}-ядра и 4 энергоэффективных \textit{E}-ядра, 
что обеспечивает поддержку 12 потоков. Распределение нагрузки между ядрами 
осуществляется аппаратным планировщиком \textit{Intel Thread Director}. 
Пиковая частота производительных ядер достигает 4.4 ГГц, объем кэш-памяти \textit{L3} 
составляет 12 МБ, а тепловой пакет (\textit{TDP}) — 45 Вт.

% NOTE:Добавить таблицу с характеристиками процессоров

\newcolumntype{Y}{>{\centering\arraybackslash}X}

\begin{table}[ht]
\centering
\caption{Сравнение характеристик}
\begin{tabularx}{\textwidth}{|c
                             |>{\centering\arraybackslash}m{5cm}
                             |Y|}
\hline
Характеристики & \textit{AMD Ryzen 7 5800H} & \textit{Intel Core i5-12450H} \\
\hline
Кодовое имя архитектуры & \textit{Cezanne-H (Zen 3)} & \textit{Alder Lake-H} \\
\hline
Физические ядра & 8 & 8 \\
\hline
Количество потоков & 16 & 12 \\
\hline
\textit{L1} кэш & 64 КБ & 80 КБ \\
\hline
\textit{L2} кэш & 512 КБ & 1.25 МБ \\
\hline
\textit{L3} кэш & 16 МБ & 12 МБ \\
\hline
Тактовая частота & 3.2 - 4.4 Ггц & 2 - 4.4 Ггц \\
\hline
Техпроцесс & 7 нм & 7 нм \\
\hline
Встроенная графика & \textit{AMD Radeon RX Vega 8} & \textit{Intel UHD Graphics} \\
\hline
Поддерживаемая память & DDR4 & DDR4, DDR5 \\
\hline
\textit{TDP} & 45 Вт & 45 Вт \\
\hline
\textit{Hyper-Threading/SMT} & + & + \\
\hline
\end{tabularx}
\end{table}

Ключевое различие заключается в подходе к реализации многоядерности. Архитектура \textit{Zen 3}
предоставляет большее количество равнозначных потоков и больший объем кэш-памяти 3 уровня. 
Архитектура \textit{Alder Lake}, в свою очередь, предлагает гибкое управление ресурсами 
через разнородные ядра. Данные архитектурные различия являются основой для 
последующего сравнительного анализа производительности.

\subsection{История, версии и достоинтсва}


\subsection{Обоснование выбора вычислительной системы}

\subsection{Анализ выбранной вычислительной системы для написания программы}
