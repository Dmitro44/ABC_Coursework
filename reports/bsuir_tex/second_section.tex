%! TeX root = main.tex

\section{Платформа программного обеспечения}

\subsection{Структура и архитектура платформы}

В качестве операционной системы для разработки и тестирования программного обеспечения была выбрана ОС \textit{Linux}. \textit{Linux} — это семейство \textit{Unix}-подобных операционных систем с открытым исходным кодом, основанных на одноименном ядре. Впервые выпущенное Линусом Торвальдсом в 1991 году, ядро \textit{Linux} стало основой для множества операционных систем, известных как дистрибутивы. 

Ключевыми особенностями \textit{Linux}, определившими его выбор, являются высокая стабильность, безопасность, гибкость настройки и модульность. Эти качества сделали его доминирующей системой в серверном сегменте, суперкомпьютерах и встраиваемых системах, а также популярной платформой для разработчиков программного 
обеспечения.

Архитектура \textit{Linux} строится на нескольких фундаментальных принципах, унаследованных от \textit{Unix}, но получивших собственное развитие:

\begin{enumerate_num_without_dot}
    \item Монолитное ядро с модульной структурой. Ядро \textit{Linux} является монолитным, 
        то есть все основные сервисы работают в едином адресном пространстве 
        (пространстве ядра), имея прямой доступ к аппаратному обеспечению. 
        Однако оно поддерживает динамическую загрузку модулей (драйверов), 
        что позволяет сохранять компактность, обеспечивая широкую поддержку 
        оборудования.

    \item Системные библиотеки.
        Они формируют ключевой промежуточный слой
        между пользовательскими приложениями и ядром системы.
        Программы, как правило, не используют системные вызовы напрямую.
        Вместо этого они обращаются к функциям из системных библиотек,
        самой известной из которых является \textit{GNU C Library (glibc)}.
        Такой подход упрощает разработку и повышает переносимость
        программного обеспечения.

    \item Многозадачность и многопользовательский режим. Система изначально проектировалась 
        для одновременной работы нескольких пользователей и параллельного выполнения 
        множества процессов в пространстве пользователя. Ядро использует механизм 
        вытесняющей многозадачности для справедливого распределения процессорного 
        времени.

    \item Стандарт иерархии файловой системы (\textit{FHS}). Структура каталогов в
        \textit{Linux} строго стандартизирована. \textit{FHS} определяет назначение основных 
        директорий, таких как /\textit{bin}, /\textit{etc}, /\textit{home} и /\textit{var}, что обеспечивает 
        предсказуемость и совместимость программного обеспечения.
    
    \item Разделение пространства. Система четко разделена на пространство ядра 
        (\textit{kernel space}) и пространство пользователя (\textit{user space}), где выполняются 
        все прикладные программы. Взаимодействие между ними происходит через строго 
        определенный интерфейс системных вызовов, что повышает стабильность и 
        безопасность системы.

    \item Широкая поддержка оборудования.
        Огромное количество драйверов для самого разного оборудования
        включено непосредственно в исходный код ядра \textit{Linux}.
        Это, в сочетании с модульной структурой,
        позволяющей подгружать только необходимые драйверы,
        обеспечивает широкую поддержку устройств <<из коробки>>.
        Активное участие сообщества и производителей оборудования
        постоянно расширяет этот список.
\end{enumerate_num_without_dot}

На рисунке \ref{fig:architech_linux} изображена архитектура операционной системы \textit{Linux}.

\begin{figure}[h!]
    \centering
    \includegraphics[width=0.8\textwidth]{fig/fundamental-architecture-of-linux.jpg}
    \caption{Архитектура операционной системы Linux}
    \label{fig:architech_linux}
\end{figure}


\subsection{История, версии и достоинства}

Возникновение операционной системы \textit{Linux} датируется 1991 годом и связано с инициативой финского исследователя Линуса Торвальдса. Изначальной целью разработки было создание \textit{UNIX}-подобного ядра, которое было бы свободно от лицензионных ограничений, присущих существовавшим на тот момент системам, таким как \textit{MINIX}. Первая официальная версия ядра, 0.01, была представлена в сентябре 1991 года.

Фундаментальным этапом в развитии \textit{Linux} стала его интеграция с набором системных утилит и библиотек проекта \textit{GNU}, инициированного Ричардом Столлманом в 1983 году. Проект \textit{GNU} ставил своей целью создание полностью свободной \textit{UNIX}-совместимой операционной системы, и к началу 1990-х годов располагал практически всеми необходимыми компонентами (т.н. \textit{userland}), за исключением низкоуровневого ядра. Объединение ядра \textit{Linux} и пользовательского окружения \textit{GNU} привела к формированию полноценной операционной системы, которую корректно именовать \textit{GNU/Linux}. Перевод проекта под юрисдикцию лицензии \textit{GNU General Public License (GPL) v2} в 1992 году стал решающим фактором, обеспечившим правовую основу для свободного распространения, модификации и коллективной разработки.

Разработка \textit{Linux} в значительной степени была вдохновлена его предшественниками, в первую очередь \textit{UNIX} и \textit{MINIX}. Система \textit{UNIX}, созданная в лабораториях \textit{Bell Labs} в конце 1960-х — начале 1970-х годов, заложила фундаментальные принципы проектирования, ставшие стандартом для современных операционных систем: иерархическая файловая система, концепция процессов и функциональный интерфейс командной строки. Её портируемость и многозадачность сделали её стандартом в академической и коммерческой среде.

Однако со временем лицензирование \textit{UNIX} становилось всё более ограничительным. Это побудило Эндрю Таненбаума в конце 1980-х годов создать \textit{MINIX} — \textit{UNIX}-подобную операционную систему на основе микроядра, предназначенную для образовательных целей. Её исходный код был доступен, но лицензия всё ещё накладывала ограничения на свободное изменение и распространение. Именно опыт работы Линуса Торвальдса с \textit{MINIX} и его желание создать по-настоящему свободную \textit{UNIX}-подобную операционную систему с открытым исходным кодом, которую он мог бы адаптировать для собственного оборудования, и послужили прямой причиной для создания ядра \textit{Linux}.

Ядро \textit{Linux} само по себе не является готовой к использованию операционной системой. Оно служит основой для дистрибутивов — комплексных программных сборок, которые включают также системные утилиты, библиотеки и прикладное программное обеспечение. Дистрибутивы, такие как \textit{Debian}, \textit{Fedora} и \textit{Ubuntu}, различаются системами управления пакетами, набором ПО и целевым назначением, предоставляя пользователям готовые к работе системы для различных задач.

Ключевые достоинства \textit{Linux} как платформы для разработки и научных 
вычислений включают следующие аспекты:

\begin{enumerate_num_without_dot}
    \item Доступность исходного кода: Лицензия \textit{GPL} гарантирует право на изучение, модификацию и распространение исходного кода. Это способствует проведению аудита безопасности, адаптации системы под специфические задачи и ускоряет цикл внедрения новых технологий.

    \item Высокая стабильность и надежность: Архитектурные принципы, унаследованные от \textit{UNIX}, обеспечивают высокую отказоустойчивость и способность к длительной непрерывной работе, что является критически важным для серверных систем и длительных вычислительных экспериментов.

    \item Модель безопасности: Система разграничения прав доступа для пользователей, групп и прочих субъектов, а также механизм изоляции процессов, значительно снижают риски, связанные с вредоносным программным обеспечением.

    \item Гибкость и модульность: Возможность детальной конфигурации всех компонентов системы, от параметров ядра до выбора графической среды, позволяет оптимизировать платформу для конкретных аппаратных ресурсов и прикладных задач.

    \item Эффективное управление ресурсами: Планировщик задач ядра \textit{Linux} и подсистема управления памятью обеспечивают эффективное использование вычислительных ресурсов, что часто выражается в более высокой производительности по сравнению с альтернативными ОС на идентичном оборудовании.

\end{enumerate_num_without_dot}

\subsection{Обоснование выбора платформы}

Для проведения исследования в качестве дистрибутива операционной системы \textit{GNU/Linux} был выбран \textit{Arch Linux}. Этот выбор обусловлен рядом ключевых преимуществ, которые делают его оптимальной платформой для проведения точного и воспроизводимого сравнительного анализа производительности.

Центральным принципом \textit{Arch Linux} является минимализм. Система поставляется в виде базового набора компонентов, предоставляя пользователю полный контроль над устанавливаемым программным обеспечением. Такой подход позволяет создать «чистую» среду для тестирования, свободную от фоновых процессов и служб, которые могли бы вносить помехи в результаты измерений. Это гарантирует, что полученные данные о времени выполнения алгоритма БПФ будут максимально точно отражать производительность самих процессоров, а не влияние стороннего ПО.

Другим важным достоинством является модель непрерывных обновлений (\textit{rolling release}). \textit{Arch Linux} предоставляет доступ к самым последним стабильным версиям программного обеспечения, включая ядро \textit{Linux}, компиляторы (\textit{GCC}, \textit{Clang}) и системные библиотеки. Использование новейшего ядра особенно актуально для данного исследования, поскольку оно включает последние оптимизации планировщика задач, что критически важно для корректной работы с гибридной архитектурой процессора \textit{Intel Core i5-12450H} и его технологией \textit{Thread Director}. Современные компиляторы, в свою очередь, могут предложить улучшенные возможности для оптимизации кода, что напрямую влияет на итоговую производительность.

Ключевым преимуществом \textit{Arch Linux} является его философия, предполагающая полный контроль над конфигурацией системы. Прозрачность и простота структуры дистрибутива позволяют точно документировать и, что более важно, воспроизводить тестовое окружение с высокой степенью достоверности. Это, в сочетании с минимализмом и доступом к новейшему программному обеспечению, обеспечивает создание изолированной и предсказуемой среды, необходимой для получения объективных и сопоставимых результатов производительности.

\subsection{Анализ операционной системы для написания программы}

Анализ операционной системы в контексте поставленной задачи заключается в определении конкретных инструментов и методологий, которые будут использованы для разработки, компиляции и выполнения программы для тестирования. Выбранная платформа, \textit{Arch Linux}, предоставляет все необходимые средства для проведения глубокого и точного исследования.

Компиляция программы будет производиться с помощью компилятора \textit{GCC} последней версии, доступной в дистрибутиве. Для достижения максимальной производительности компиляция будет выполняться с флагами, активирующими высокий уровень оптимизации, такие как \textit{-O3} и \textit{-march=native}. Флаг \textit{-march=native} заставляет компилятор генерировать код, оптимизированный под специфический набор инструкций конкретного процессора, на котором производится сборка. Это позволит в полной мере задействовать архитектурные преимущества каждого из тестируемых \textit{CPU}.

% Ключевым аспектом анализа станет управление выполнением потоков. Операционная система \textit{Linux} предоставляет мощные механизмы для привязки процессов и потоков к конкретным физическим ядрам процессора (\textit{CPU affinity}). С помощью утилиты \textit{taskset} или системных вызовов, таких как \textit{sched\_setaffinity}, можно будет изолированно измерять производительность:
%
% \begin{enumerate_num_without_dot}
%     \item Производительных (\textit{P-cores}) и энергоэффективных (\textit{E-cores}) ядер процессора \textit{Intel Core i5-12450H}.
%     \item Различного числа ядер (от 1 до 8) процессора \textit{AMD Ryzen 7 5800H}.
% \end{enumerate_num_without_dot}

Такой подход позволит детально изучить вклад каждого типа ядер в общую производительность и оценить эффективность планировщика задач.

Для получения более глубоких данных о производительности, помимо простого измерения времени выполнения, планируется использование утилиты \textit{perf}. Этот инструмент позволяет собирать данные с аппаратных счетчиков производительности процессора, такие как количество выполненных инструкций, промахи кэша и неверно предсказанные переходы. Анализ этих метрик даст возможность сделать более обоснованные выводы о причинах наблюдаемых различий в производительности.

Таким образом, \textit{Arch Linux} предоставляет полный набор инструментов для разработки, низкоуровневой оптимизации и детального анализа производительности, что делает его идеальной средой для решения задач данного исследования.

